\chapter{Logica}
\label{Capitolo 2}
\emph{Si ringrazia
  \MYhref{https://github.com/bigboss98/Appunti-1/tree/master/Primo Anno/Fondamenti}{Marco
    Natali}
  \footnote{Link al repository:
    https://github.com/bigboss98/Appunti-1/tree/master/Primo Anno/Fondamenti} per questo
  ripasso}.\\\\   
La logica è lo studio del ragionamento e dell’argomentazione e, in particolare,
dei procedimenti inferenziali, rivolti a chiarire quali	procedimenti di pensiero
siano validi e quali no. Vi sono molteplici tipologie di logiche, come ad
esempio la logica classica e le logiche costruttive, tutte accomunate dall'essere
composte da 3 elementi: 
% Elementi di una Logica
\begin{itemize}
  \item \textbf{Linguaggio}: insieme di simboli utilizzati nella Logica per
  definire le cose.
  \item \textbf{Sintassi}: insieme di regole che determina quali elementi
  appartengono o meno al linguaggio.
  \item \textbf{Semantica}: permette di dare un significato alle formule del
  linguaggio e determinare se rappresentano o meno la verità.
\end{itemize}

\section{Logica proposizionale}
Ci occupiamo della \textit{logica classica} che si compone in \textit{logica proposizionale} 
e \textit{logica predicativa}.
La logica proposizionale è quindi un tipo di logica classica che presenta come
caratteristica principale quella di essere un linguaggio limitato, ovvero caratterizzato dal poter
esprimere soltanto proposizioni senza possibilità di estensione ad una
classe di persone.
\newpage
\subsection{Sintassi}
Il linguaggio di una logica proposizionale è composto dai seguenti elementi:

% Elementi linguaggio logica proposizionale
\begin{itemize}
  \item Variabili Proposizionali atomiche (o elementari): $P,Q,R,p_i, \dots$. 
  \item Connettivi Proposizionali: $\land, \lor, \neg, \implies, \siff$
  \item Simboli Ausiliari: ``(`` e ``)'' (detti delimitatori)
  \item Costanti: $T$ (\textit{True, Vero, $\top$}) e $F$ (\textit{False, Falso,
    $\bot$})
\end{itemize}

La sintassi di un linguaggio è composta da una serie di formule ben
formate ($FBF$) che possono essere "Composte" o meno, e sono definite induttivamente nel seguente modo:
% definizione formule ben formate
\begin{enumerate}
  \item Le costanti e le variabili proposizionali: $\top,\bot,p_i\in FBF$.
  \item Se $A$ e $B \in FBF$ allora $(A \land B)$,$(A \lor B)$,$(\neg A)$,$(A
  \simplies B)$, $(A \siff B)$ sono delle formule ben formate.
  \item nient'altro è una formula
\end{enumerate}

\textbf{In una formula ben formata le parentesi sono bilanciate.}

\begin{esempio}
  Vediamo degli esempi:
  \begin{itemize}
    \item $(P \land Q) \in FBF$  è una formula ben formata\newline
    \item $(PQ \land R) \not \in FBF$ in quanto non si rispetta la sintassi del
    linguaggio 
    definita. 
  \end{itemize}
\end{esempio}

% Definizione delle sottoformule
\begin{definizione}

  Sia $A \in FBF$, l'insieme delle sottoformule di $A$ è definito come segue:
  \begin{enumerate}
    \item Se $A$ è una costante o variabile proposizionale allora A stessa è la
    sua 
    sottoformula.
    \item Se $A$ è una formula del tipo $(\neg A')$ allora le sottoformule di A
    sono 
    A stessa e le sottoformule di $A'$; 
    $\neg$ è detto connettivo principale e $A'$ sottoformula immediata di A.
    \item Se $A$ è una formula del tipo $B \circ C$, allora le sottoformule di A
    sono A stessa 
    e le sottoformule di B e C; $\circ$ è il connettivo principale e B e C sono
    le due sottoformule immediate di A. 
  \end{enumerate}

\end{definizione}
È possibile ridurre ed eliminare delle parentesi attraverso l'introduzione della
precedenza tra gli operatori, definita come segue: 
$$
\neg, \land, \lor, \simplies,\siff
$$

In assenza di parentesi una formula va parentizzata privilegiando le
sottoformule 
i cui connettivi principali hanno la precedenza più alta.\newline
In caso di parità di precedenza vi è la convenzione di associare da destra a
sinistra. Segue un esempio:
$$
\neg A \land (\neg B \simplies C) \lor D 
\hbox{ diventa }
((\neg A) \land ((\neg B) \simplies C) \lor D)
$$
\subsubsection{Albero Sintattico}
% Definizione di albero sintattico
\begin{definizione}
  Un albero sintattico $T$ è un albero binario coi nodi etichettati da simboli
  di $L$, che rappresenta la scomposizione di una formula ben formata $X$
  definita 
  come segue: 
\end{definizione}
\begin{enumerate}
  \item Se $X$ è una formula atomica, l'albero binario che la rappresenta è
  composto 
  soltanto dal nodo etichettato con $X$
  \item Se $X = A \circ B$, $X$ è rappresentata da un albero binario che ha la
  radice 
  etichettata con $\circ$, i cui figli sinistri e destri sono la
  rappresentazione di $A$ e $B$ 
  \item Se $X = \neg A$, $X$ è rappresentato dall'albero binario con radice
  etichettata 
  con $\neg$, il cui figlio è la rappresentazione di $A$
\end{enumerate}

Poiché una formula è definita mediante un albero sintattico, le proprietà di una
formula 
possono essere dimostrate mediante induzione strutturale sulla formula, ossia
dimostrare 
che la proprietà di una formula soddisfi i seguenti 3 casi:
\begin{itemize}
  \item è verificata la proprietà per tutte le formule atomo $A$
  \item supposta verifica la proprietà per $A$, si verifica che la proprietà è
  verificata per $\neg A$ 
  \item supposta la proprietà verificata per $A_1$ e $A_2$, si verifica che la
  proprietà è verifica per $A_1 \circ A_2$, per ogni connettivo $\circ$.
\end{itemize}
\newpage
\subsection{Semantica}
La semantica di una logica consente di dare un significato e un'interpretazione
alle formule del Linguaggio.\newline
\begin{definizione}
  Sia data una formula proposizionale $P$ e sia ${P_1,\dots,P_n}$, l'insieme
  degli 
  atomi che compaiono nella formula $A$. Si definisce come
  \emph{interpretazione} una 
  funzione $v:\{P_1,\dots,P_n\} \mapsto \{T,F\}$ che attribuisce un valore di
  verità 
  a ciascun atomo della formula $A$.
  \\
  $v:P\to\{0,1\}$ è un'\textbf{assegnazione booleana} 
\end{definizione}

I connettivi della Logica Proposizionale hanno i seguenti valori di verità:
% Tabella di Verità degli operatori
\[
  \begin{array}{ccccccc}
    \toprule
    \text{A} & \text{B} & A \land B & A \lor B & \neg A & A \simplies B & A
                                                                          \siff
                                                                          B \\
    \midrule
    F & F & F & F & T & T & T \\
    F & T & F & T & T & T & F \\
    T & F & F & T & F & F & F \\
    T & T & T & T & F & T & T \\
    \bottomrule
  \end{array}
\]
Essendo ogni formula $A$ definita mediante un unico albero sintattico,
l'interpretazione $v$ 
è ben definita e ciò comporta che data una formula $A$ e un'interpretazione
$v$, 
eseguendo la definizione induttiva dei valori di verità, si ottiene un unico
$v(A)$. 

% Tipologie di formule
\begin{definizione}
  Una formula nella logica proposizionale può essere di diversi tipi:
  \begin{itemize}
    \item \textbf{Valida o Tautologica:} la formula è soddisfatta da qualsiasi
    valutazione della Formula 
    \item \textbf{Soddisfacibile NON Tautologica:} la formula è soddisfatta da
    qualche valutazione 
    della formula ma non da tutte.
    \item \textbf{Falsificabile:} la formula non è soddisfatta da qualche
    valutazione della formula. 
    \item \textbf{Contraddizione:} la formula non viene mai soddisfatta
  \end{itemize}
\end{definizione}

\begin{teorema}
  Si ha che:
  \begin{itemize}
    \item $A$ è una formula valida se e solo se $\neg A$ è insoddisfacibile.
    \item $A$ è soddisfacibile se e solo se $\neg A$ è falsificabile
  \end{itemize}
\end{teorema}


\subsubsection{Modelli e decidibilità}
Si definisce \emph{modello}, indicato con $M \models A$, tutte le valutazioni
booleane 
che rendono vera la formula $A$. Quindi un modello riguarda i valori di verità di tutte le proposizioni atomiche che, insieme, soddisfano A.
Si definisce \emph{contromodello}, indicato con $\not\models$, tutte le
valutazioni booleane 
che rendono falsa la formula $A$.

La logica proposizionale è decidibile (posso sempre verificare il significato di
una formula). 
Esiste infatti una procedura effettiva che stabilisce la validità o no di una
formula, o se questa 
ad esempio è una tautologia.
In particolare il verificare se una proposizione è tautologica o meno è
l’operazione di decidibilità principale che si svolge nel calcolo
proposizionale. 

\begin{definizione}
  Se $M \models A$ per tutti gli $M$, allora $A$ è una tautologia e si indica
  $\models A$.
\end{definizione}

\begin{definizione}
  Se $M \models A$ per qualche $M$, allora $A$ è soddisfacibile.
\end{definizione}

\begin{definizione}
  Se $M \models A$ non è soddisfatta da nessun $M$, allora $A$ è
  insoddisfacibile.
\end{definizione}
\subsection{Equivalenze Logiche}
\begin{definizione}
  Date due formule $A$ e $B$, si dice che $A$ è \emph{logicamente equivalente} a
  $B$, 
  indicato con $A \equiv B$, se e solo se per ogni interpretazione $v$ risulta
  $v(A) = v(B)$. 
\end{definizione}

Nella logica proposizionale sono definite le seguenti equivalenze logiche,
indicate con $\equiv$: 
\begin{enumerate}
  \item \textbf{Idempotenza:}
  \begin{align*}
    A \lor A  \equiv  A \\
    A \land A  \equiv  A \\
  \end{align*}
  \item \textbf{Associatività:}
  \begin{align*}
    A \lor (B \lor C) \equiv  (A \lor B) \lor C \\
    A \land (B \land C)  \equiv  (A \land B) \land C
  \end{align*}
  \item \textbf{Commutatività:}
  \begin{align*}
    A \lor B  \equiv  B \lor A \\
    A \land B  \equiv  B \land A
  \end{align*}
  \item \textbf{Distributività:}
  \begin{align*}
    A \lor (B \land C)  \equiv & (A \lor B) \land (A \lor C)\\
    A \land (B \lor C)  \equiv & (A \land B \lor (A \land C)
  \end{align*}
  \item \textbf{Assorbimento:}
  \begin{align*}
    A \lor (A \land B)  \equiv  A
    A \land (A \lor B)  \equiv  A
  \end{align*}
  \item \textbf{Doppia negazione:}
  \begin{equation*}
    \neg \neg A \equiv A
  \end{equation*}
  \item\textbf{Leggi di De Morgan:}
  \begin{align*}
    \neg (A \lor B)  \equiv  \neg A \land \neg B \\
    \neg(A \land B)  \equiv  \neg A \lor \neg B
  \end{align*}
  \item \textbf{Terzo escluso:}
  \begin{equation*}
    A \lor \neg A \equiv T
  \end{equation*}
  \item \textbf{Contrapposizione:}
  \begin{equation*}
    A \simplies B \equiv \neg B \simplies \neg A
  \end{equation*}
  \item \textbf{Contraddizione}
  \begin{equation*}
    A \land \neg A \equiv F
  \end{equation*}
\end{enumerate}
\subsubsection{Completezza di insiemi di Connettivi}
Un insieme di connettivi logici è completo se mediante i suoi connettivi si può
esprimere un qualunque altro connettivo.
Nella logica proposizionale valgono anche le seguenti equivalenze, utili per
ridurre il linguaggio: 

\[(A \simplies B)  \equiv  (\neg A \lor B) \]
\[(A \lor B)  \equiv  \neg(\neg A \land \neg B) \]
\[(A \land B)  \equiv  \neg(\neg A \lor \neg B) \]
\[(A \siff B) \equiv  (A \simplies B) \land (B \simplies A) \]

L'insieme dei connettivi $\{ \neg,\lor,\land \}$, $\{ \neg,\land \}$ e $\{
\neg,\lor \}$ sono completi.
