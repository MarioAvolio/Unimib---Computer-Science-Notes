\chapter{Concept Learning}
%%%%
In questo capitolo studieremo metodi di apprendimento che possono mettere a frutto la \textbf{conoscenza a priori}. In particolare essa verrà rappresentata come un insieme di teorie generali di \href{https://it.wikipedia.org/wiki/Linguaggio_del_primo_ordine}{logica del primo ordine\footnote{Da Wikipedia: Nella logica matematica il linguaggio del primo ordine è un linguaggio formale che serve per gestire meccanicamente enunciati e ragionamenti che coinvolgono i connettivi logici, le relazioni e i quantificatori "per ogni ..." (∀) ed "esiste..." (∃). L'espressione "del primo ordine" indica che c'è un insieme di riferimento e i quantificatori possano riguardare solo gli elementi di tale insieme e non i sottoinsiemi; ad esempio si può dire "per tutti gli x elementi dell'insieme vale P(x)" ma non si può dire "per tutti i sottoinsiemi A vale P(A)" (le teorie in cui ci sono quantificatori che spaziano sui sottoinsiemi dell'insieme di riferimento sono dette invece del secondo ordine).}}.
\section{Una formulazione logica dell'apprendimento}
Nel capitolo precedente abbiamo definito l'apprendimento induttivo puro come il processo mediante il quale si cerca un'ipotesi che concorda con gli esempi osservati. Ora specializzeremo tale definizione per il caso in cui l'ipotesi sia rappresentata da un insieme di formule logiche. Questo approccio consente di creare delle ipotesi in maniera incrementale; inoltre consente di sfruttare la conoscenza pregressa, perché le formule già note possono aiutare la classificazione di nuovi esempi. Il \textbf{concept learning} è la ricerca, nello spazio delle ipotesi, di funzioni che assumano valori all'interno di $\{0, 1\}$. In altre parole si parla di funzioni che hanno come dominio lo \textbf{spazio delle ipotesi} e come codominio $\{0, 1\}$:
\[f:H\to\{0, 1\}\]
Volendo si possono usare insiemi e non funzioni.\\
\subsection{Esempi e Ipotesi}
Un esempio è un oggetto descritto da una formula logica e gli attributi sono predicati unari. L'intero insieme di addestramento sarà quindi espresso da una congiunzione di tutte le descrizioni e delle formule di classificazione. Lo scopo dell'apprendimento induttivo in ambito logico è trovare un'espressione equivalente al \textbf{predicato obiettivo} (la nostra ipotesi) utilizzabile per classificare correttamente gli esempi.
Si cerca quindi con opportune procedure la miglior ipotesi che si adatta
meglio al concetto implicato dal \textit{training set}.
In questo contesto si cerca di capire quale funzione booleana è adatta al mio addestramento. In altre parole si cerca di apprendere un'ipotesi booleana partendo da esempi di training composti da input e output (etichetta target + label) della funzione. \\
Nel concept learning un'ipotesi è un insieme di valori di attributi e ogni valore può essere:
\begin{itemize}
  \item Specificato
  \item Indifferente: che si indica con ``?'', e che può assumere qualsiasi
  valore. Avere un'ipotesi con tutti i valori del vettore pari a ``?'' implica
  avere l'ipotesi più \textbf{generale}, avendo classificato tutte le istanze solo come
  esempi positivi. 
  \item Nulla: si indica con $\emptyset$. Avere un'ipotesi con tutti i valori
  del vettore pari a $\emptyset$ implica avere l'ipotesi più \textbf{specifica}, avendo
  classificato tutte le istanze solo come esempi negativi. 
\end{itemize}
\begin{esempio}
  Vediamo quindi la rappresentazione di una ipotesi (ipotizzando di avere a che
  fare con solo 4 attributi $A_i$, sempre in prospettiva booleana):
  \[h=\langle 0, 1, ?, 1\rangle = \langle A_1=0, A_2=1, A_3=\,?, A_4=1\rangle\]
  nella realtà, grazie al ``?'' riferito all'istanza, l'ipotesi $h$ è un insieme
  di due ipotesi: 
  \[h\in\{(0, 1, 0, 1),\,(0, 1, 1, 1)\}\]
  passando quindi da una notazione per ipotesi a una insiemistica.\\
  Ricordiamo inoltre che lo spazio delle istanze $X$, dal punto di vista
  insiemistico è:
  \[X=\{(x_1, x_2, x_3, x_3): \,\, x_1\in A_1, x_2\in A_2, x_3\in A_3, x_4\in A_4\}\]
\end{esempio}
\subsubsection{Soddisfacimento e Consistenza}
Ogni ipotesi predice che un certo insieme di esempi e, precisamente quelli che la soddisfano, saranno anche esempi del \textbf{predicato obiettivo}. Questo insieme è chiamato \textbf{estensione del predicato}. \begin{definizione}
  Due ipotesi con estensioni differenti sono quindi logicamente inconsistenti l'una con l'altra, perché le loro previsioni sono discorsi su almeno un esempio.
\end{definizione}
Sia \textit{c(x)} una \textbf{funzione target}, ovvero la funzione che, all'interno di un certo \textit{training set} $D$, esprime il valore di verità di un'istanza.
\begin{definizione}
Si dice che un'istanza $x$ \textbf{soddisfa} i vincoli di $h$ sse $h$ classifica $x$ come
esempio positivo:
\[h(x)=1\]  
Ciò è \textbf{indipendente} dal valore della funzione target associato ad x.
\end{definizione}

\begin{definizione}
 Si dice \textbf{consistente} un'ipotesi $h\in H$ nei confronti di un determinato insieme addestramento sse essa dev'essere consistente con ogni esempio di tale insieme. Ovvero: 
\[h(x)=c(x),\,\forall x\in D\]  
\end{definizione}

Si evince la teoria delle \textbf{ipotesi di apprendimento induttivo} che dice che se la mia $h$ approssima bene nel \textit{training set} allora approssima bene su tutti gli esempi non ancora osservati.\\
Il concept learning è quindi una ricerca del \textit{fit} migliore.
\subsubsection{Falso negativo e Falso positivo}
Se un esempio è un falso positivo o un falso negativo per un'ipotesi, l'esempio e l'ipotesi sono logicamente inconsistenti. Presumendo che l'esempio rappresenti l'osservazione corretta di un fatto, l'ipotesi può quindi essere scartata. Possiamo quindi caratterizzare l'apprendimento induttivo in ambito logico come un processo di \textbf{eliminazione graduale} delle ipotesi che si rivelano inconsistenti con gli esempi.
\begin{itemize}
    \item Un esempio può essere un \textbf{falso negativo} per l'ipotesi se essa predice che dovrebbe essere negativo ma, in effetti, è positivo. L'ipotesi e l'esempio sono quindi logicamente inconsistenti.
    \item Un esempio può essere un \textbf{falso positivo} per l'ipotesi se essa dice che dovrebbe essere positivo ma, in effetti, è negativo.
\end{itemize}
\section{Generalità delle ipotesi}
\begin{definizione}
  Date $h_j, h_k\in H$ booleane e definite su $X$. Si ha che $h_j$ è \textbf{più
    generale o uguale a} $h_k$ (e si scrive con $h_j\geq h_k$) sse:
  \[(h_k(x)=1)\longrightarrow (h_j(x)=1),\,\,\forall x\in X\]
  \textbf{Si impone quindi un ordine parziale}.\\
  Si ha che $h_j$ è \textbf{più generale di} $h_k$ (e si scrive con $h_j> h_k$)
  sse:
  \[(h_j\geq h_k)\land (h_k\not\geq h_j)\]
  Riscrivendo dal punto di vista insiemistico si ha che $h_j$ è \textbf{più
    generale o uguale a} $h_k$ sse:
  \[h_j\supseteq h_k\]
  e che è \textbf{più generale di} $h_k$ sse:
  \[h_j\supset h_k\]
  Dal punto di vista logico si ha che $h_j$ è \textbf{più generale di} $h_k$ sse
  impone meno vincoli di $h_k$( esempio \ref{fig:eulero})
\end{definizione}
\subsection{Esempi}
\begin{esempio}
  Facciamo un esempio ``giocattolo'' di situazione di \textbf{concept
    learning}.\\ 
  La \textbf{funzione target} (o target concept), ovvero la domanda che ci si pone, è:\\
  \begin{center}
    \textit{In quali tipologie di giorni $A$ apprezza fare sport acquatici?}
  \end{center}
  Abbiamo quindi una serie di attributi per il meteo con i vari valori che
  possono assumere:
  \begin{table}[H]
    \centering
    \begin{tabular}[H]{|c|c|}
      \hline
      \textbf{Attributo} & \textbf{\textit{Possibili valori}}\\
      \hline
      sky & \textit{Sunny, Cloudy, Rainy}\\
      temp & \textit{Warm, Cold}\\
      humid & \textit{Normal, High}\\
      wind & \textit{Strong, Weak}\\
      water & \textit{Warm, Cold}\\
      forecast & \textit{Same, Change}\\
      \hline
    \end{tabular}
  \end{table}
  In ottica \textit{concept learning} si cerca quindi la \textbf{funzione
    target} $enjoySport$ (che sarebbe la nostra funzione $c$): 
  \[enjoySport:X\to\{0, 1\}\]
  Vediamo quindi anche un esempio di \textit{training set} $D$ (per praticità i
  valori degli attributi sono indicati con la sola iniziale):
  \begin{table}[H]
    \centering
    \begin{tabular}[H]{|c|c|c|c|c|c|c|}
      \hline
      \textbf{sky} & \textbf{temp} & \textbf{humid} & \textbf{wind} &         
      \textbf{water} & \textbf{forecast} & \textbf{enjoySport}\\
      \hline
      S & W & N & S & W & S & \color{darkgreen} yes\\
      S & W & H & S & W & S & \color{darkgreen} yes\\
      R & C & H & S & W & C & \color{red} no\\
      S & W & H & S & C & C & \color{darkgreen} yes\\
      \hline
    \end{tabular}
  \end{table}
  Dove nella colonna finale si ha la risposta booleana al problema, è infatti
  l'\textbf{etichetta target} (valore della funzione target per un determinato attributo).\\
  Bisogna quindi cercare un'ipotesi $h$ tale che $h(x)=c(x)$,
  per tutte le istanze nel \textit{training set}.\\
  Un'ipotesi $h$, anch'essa rappresentata come vettore, sarà quindi una
  congiunzione $\land$ di vincoli di valore sugli attributi.
  \label{es:tab}
\end{esempio}
\begin{esempio}
  Vediamo un esempio anche relativo alle nozioni di \textbf{più generale di}.\\
  Si hanno due ipotesi:
  \[h_J=\langle S,?,?,?,?,?\rangle\]
  \[h_k=\langle S,?,?, S,?,?\rangle\]
  e quindi si ha che:
  \[h_J\geq h_k \mbox{\textnormal{ovvero} }h_j(x)=1\implies h_k(x)=1\]
  infatti un'istanza positiva per $h_j$ è sicuramente positiva anche per $h_k$,
  in quanto $h_k$ ha un vincolo più restrittivo sul quarto attributo, che deve
  assumere il valore $S$, mentre il quarto attributo di $h_j$ può assumere
  qualsiasi valore. Questo aspetto si potrebbe rappresentare con un
  \textbf{diagramma di Eulero-Venn} dal punto di vista insiemistico (con il
  ``cerchio'' di $h_j$ che conterrebbe quello di $h_k$ (essendo un insieme più
  esterno e quindi più generale), nello spazio delle istanze).
  \begin{figure}
    \centering
    
    \psscalebox{1 1} % Change this value to rescale the drawing.
    {
      \begin{pspicture}(0,-2.3375487)(4.0, 2.3375487)
        \definecolor{colour0}{rgb}{0.99607843, 0.30980393, 0.30980393}
        \definecolor{colour1}{rgb}{0.25490198, 0.41960785, 0.94509804}
        \definecolor{colour2}{rgb}{0.4117647, 0.6784314, 0.31764707}
        \psframe[linecolor=colour0, linewidth=0.04, dimen=outer]
        (4.0, 1.6624511)(0.0,-2.3375487)
        \pscircle[linecolor=colour1, linewidth=0.04, dimen=outer]
        (2.0,-0.33754882){1.6}
        \rput{-271.73}(1.0203394,-1.7267202){
          \psellipse[linecolor=colour2, linewidth=0.04, dimen=outer]
          (1.4,-0.33754882)(0.6, 0.4)}
        \psdots[linecolor=black, dotsize=0.1](2.4,-0.33754882)
        \psdots[linecolor=black, dotsize=0.1](2.4,-0.7375488)
        \psdots[linecolor=black, dotsize=0.1](1.6,-1.1375488)
        \psdots[linecolor=black, dotsize=0.1](1.2,-0.33754882)
        \psdots[linecolor=black, dotsize=0.1](0.4, 1.2624512)
        \psdots[linecolor=black, dotsize=0.1](3.6, 1.2624512)
        \psdots[linecolor=black, dotsize=0.1](3.6, 0.8624512)
        \psdots[linecolor=black, dotsize=0.1](3.2,-1.9375489)
        \psdots[linecolor=black, dotsize=0.1](0.4,-1.9375489)
        \psdots[linecolor=black, dotsize=0.1](0.8,-1.9375489)
        \psdots[linecolor=black, dotsize=0.1](2.8, 0.46245116)
        \psdots[linecolor=black, dotsize=0.1](3.2,-0.33754882)
        \psdots[linecolor=black, dotsize=0.1](3.2,-0.33754882)
        \psdots[linecolor=black, dotsize=0.1](2.8,-0.7375488)
        \psdots[linecolor=black, dotsize=0.1](2.0, 0.8624512)
        \rput[bl](0.1, 1.8624511){Spazio delle ipotesi $X$}
        \rput[bl](0.28, 0.46245116){$h_j$}
        \rput[bl](1.7, 0.062451173){$h_k$}
      \end{pspicture}
    }
    \label{fig:eulero}
    \caption{Rappresentazione di \textbf{più generale di} con i diagrammi di
      Eulero-Venn, dove si nota come, in $X$, $h_j\geq h_k$} 
  \end{figure}
\end{esempio}
\begin{esempio}
  Consideriamo un'ipotesi $h$, con quattro attributi, dal punto di vista
  insiemistico: 
  \[h\in\{( 0, 1, 1, 1)\,(0, 1, 0, 0)\}\]
  e ci si chiede se $h\in H$.\\
  In primis possiamo ``comprimere'' questa rappresentazione insiemistica in:
  \[h=\langle 0, 1, ?, ?\rangle\]
  $h$ ora è rappresentata come tipicamente fatto nel \textit{concept
    learning}.\\
  Quindi $h\in H$, avendo un'ipotesi con quattro attributi. 
\end{esempio}
\begin{esempio}
  Consideriamo un concetto $c$, con cinque attributi, dal punto di vista
  insiemistico:
  \[c\in
    \begin{cases}
      \begin{rcases}
        (0, 1, 0, 1, 0),\\
        (0, 1, 1, 1, 0),\\
        (0, 1, 0, 1, 1),\\
        (0, 1, 1, 1, 1)
      \end{rcases}
    \end{cases}
  \]
  e ci si chiede se $c\in H$, quindi se la strategia è di ricerca è buona esso
  può essere ritrovato (altrimenti nessuna strategia lo potrebbe riconoscere).\\
  Studiando $c$ otteniamo che:
  \[c=\langle 0, 1,?, 1,?\rangle\]
  \textbf{Usiamo la stessa rappresentazione usata per le ipotesi}
\end{esempio}
\begin{esempio}
  Vediamo un esempio di studio dello spazio delle istanze $X$.\\
  Considero che le istanze sono specificate da due attributi: $A_1=\{0, 1, 2\}$ e
  $A_2=\{0, 1\}$.
  Quindi, sapendo che $X=A_1\times A_2$ (con $\times$ \textbf{prodotto
    cartesiano}), ho che: 
  \[|X|=|A_1|\times |A_2|\]
  e quindi in questo caso $|X|=6$
\end{esempio}
\begin{esempio}
  Vediamo un esempio di studio del numero di concetti.\\
  in $X$ abbiamo 3 attributi, ciascuno con 3 possibili valori:
  $A_i=\{0, 1, 2\},\,\, i= 1, 2, 3$
  Abbiamo già visto come calcolare $|X|$ e quindi sappiamo che:
  \[|X|=3^3=27\]
  
  Sappiamo che un concetto $c$ è un sottoinsieme di $X$, quindi, chiamato
  $C=\{c_i\subseteq X\}$ l'insieme di tutti i concetti so che la sua cardinalità
  è pari alla cardinalità dell'\textbf{insieme delle parti}\footnote{In matematica, dato un insieme S, l'insieme delle parti di S, scritto P (S), è l'insieme di tutti i possibili sottoinsiemi di S. Questa collezione d'insiemi viene anche detta insieme potenza di S o booleano di S.} di $X$:
  \[|C|=|\mathcal{P}(X)|=2^{|X|}=2^{27}\]
\end{esempio}
\begin{esempio}
  Vediamo un esempio di studio del numero delle ipotesi, ovvero si studia la
  cardinalità dello spazio delle ipotesi (che altro non è che il numero di
  differenti combinazioni di tutti i possibili valori per ogni possibile
  attributo).\\
  Bisogna notare che l'uso di $\emptyset$ come valore di un attributo in
  un'ipotesi rende tale ipotesi \textbf{semanticamente equivalente} a qualsiasi
  altra che contenga un $\emptyset$ (anche non per lo stesso attributo). Inoltre
  tutte queste sono ipotesi che rifiutano tutto. Tutte queste ipotesi conteranno
  come un unico caso nel conteggio delle ipotesi.\\
  Quindi per conteggiare tutte ipotesi \textbf{semanticamente differenti} dovrò
  fare (indicando con $|A|$ il numero di valori possibili per l'attributo $A$):
  \[|H|_{sem}=1+\prod_{i}(|A_i|+1)\]
  quindi moltiplicare tutti il numero di valori di ogni attributo più 1
  (indicante ``?'' e che quindi va conteggiato come possibile valore per ogni
  attributi) e sommare 1 (indicante $emptyset$ che va conteggiato solo una volta
  per il discorso fatto sopra in merito all'equivalenza semantica di tali
  ipotesi) a questo risultato finale.\\
  Qualora volessi conteggiare tutte ipotesi \textbf{sintatticamente differenti}
  dovrò contare $\emptyset$ come ipotetico valore per ogni attributo e quindi:
  \[|H|_{sint}=\prod_{i} (|A_i|+2)\]
\end{esempio}
\newpage
%----------------------------------------------------------------------------------------
%	SECTION 
%----------------------------------------------------------------------------------------
\section{Algoritmi per la creazione d'Ipotesi}
\subsection{Algoritmo find-S}
Parliamo ora dell'algoritmo \textbf{Find-S}. Questo algoritmo permette di
partire dall'ipotesi più specifica (attributi nulli, indicati con
$\emptyset$) per poi generalizzarla, 
trovando a ogni passo un'ipotesi più adatta e consistente con il training
set $D$. L'ipotesi in uscita sarà anche consistente con gli esempi negativi
dando prova che il target è effettivamente in $H$. Con questo algoritmo non si
può dimostrare di aver trovato l'unica ipotesi consistente con gli esempi e,
ignorando gli esempi negativi non posso capire se $D$ contiene dati
inconsistenti. Inoltre non ho l'ipotesi più generale.
\begin{algorithm}[H]
  \begin{algorithmic}
    \Function{findS}{}
    \State $h\gets \mbox{l'ipotesi più specifica in } H$
    \For {\textit{ogni istanza di training positiva $x$}}
    \For {\textit{ogni vincolo di attributo $a_i$ in $h$}}
    \If {il vincolo di attributo $a_i$ in $h$ è soddisfatto da $x$}
    \State \textit{non fare nulla}
    \Else
    \State \textit{sostituisci $a_i$ in $h$ con il successivo vincolo più}
    \State \textit{generale che è soddisfatto da $x$}
    \EndIf
    \EndFor
    \EndFor
    \Return \textit{ipotesi $h$}
    \EndFunction
  \end{algorithmic}
  \caption{Algoritmo Find-S}
\end{algorithm}
Si ha che il \textbf{bias induttivo} permette di studiare anche esempi non
visti nel training set, tuttavia è una situazione rara da analizzare a causa
dell'inconsistenza stessa degli esempi e del loro rumore.
\begin{esempio}
  Prendiamo il seguente \textit{training set} $D$:
  \begin{table}[H]
    \centering
    \begin{tabular}[H]{c|cccc|c}
      esempio & $A_1$  & $A_2$  & $A_3$  & $A_4$ & label\\
      \hline
      $x_1$ & 1 & 0 & 1 & 0 & 1\\
      $x_2$ & 0 & 1 & 0 & 0 & 0\\
      $x_3$ & 1 & 0 & 1 & 1 & 0\\
      $x_4$ & 0 & 0 & 0 & 0 & 1\\
      $x_5$ & 0 & 0 & 1 & 0 & 1\\
    \end{tabular}
  \end{table}
  Applico \textbf{find-S}, che ad ogni iterazione cerca di generalizzare l'ipotesi.\\
  Parto dall'ipotesi più specifica in assoluto:
  \[S=\{\langle\emptyset,\emptyset,\emptyset,\emptyset\rangle\}\]
  Si presenta il primo esempio $x_1=(1, 0, 1, 0)$ che ha label 1, quindi
  $t(x_1)=1$. Il primo valore non soddisfa il vincolo imposto da
  $S$, quindi in $S$ faccio il \textit{replace} del valore del primo attributo
  di $x_1$ con quello di $S$. Visto che $S$ è però l'ipotesi più specifica dovrò
  farlo per tutti i vincoli, ottenendo che $S$ diventa, di fatto, una copia di
  $x_1$: 
  \[S=\{\langle 1, 0, 1, 0\rangle\}\]
  Si presenta il secondo esempio $x_2=(0, 1, 0, 0)$ che ha label 0, quindi
  $t(x_2)=0$. Essendo un esempio negativo viene ignorato dall'algoritmo,
  lasciando inalterato $S$.\\
  Si presenta il terzo esempio $x_3=(1, 0, 1, 1)$ che ha label 0, quindi
  $t(x_3)=0$. Essendo un esempio negativo viene ignorato dall'algoritmo,
  lasciando inalterato $S$.\\
  Si presenta il quarto esempio $x_4=(0, 0, 0, 0)$ che ha label 1, quindi
  $t(x_4)=1$. Controllo quindi con $S$ i vari valori. Dove si ha lo stesso
  valore lo tengo in $S$, altrimenti pongo il caso generico ``?'' (sempre
  nell'ottica voler generalizzare il più possibile), in quanto ho
  due esempi che mi dicono che avere, per quell'attributo, 1 o 0 va ugualmente
  bene. Ottengo:
  \[S=\{\langle ?, 0,?, 0\rangle\}\]
  Si presenta il quinto esempio $x_5=(0, 0, 1, 0)$ che ha label 1, quindi
  $t(x_5)=1$. Procedo come sopra e ottengo, infine:
  \[S=\{\langle ?, 0,?, 0\rangle\}\]
  Si arriva quindi a dire che $S=\{\langle ?, 0,?, 0\rangle\}$ è la nuova ipotesi
  più specifica che verrà restituita dall'algoritmo \textbf{find-S}.
\end{esempio}
\begin{esempio}
  Si ha la richiesta opposta allo scorso esempio. \\
  Data l'ipotesi:
  \[S=\{\langle ?, 0,?,?\rangle\}\]
  Si trovino 5 esempi (3 positivi e due negativi) che portino, secondo
  \textbf{find-S}, a restituire quell'ipotesi.\\
  Parto sempre da:
  \[S=\{\langle\emptyset,\emptyset,\emptyset,\emptyset\rangle\}\]
  e mi invento gli esempi.\\
  Un primo è $x_1=(1, 0, 1, 0)$ (dato che il secondo attributo dell'ipotesi fornita ha sicuramente valore 0)
  tale che $t(x_1)=1$. Raggiungo quindi: 
  \[S=\{\langle 1, 0, 1, 0\rangle\}\]
  ora me ne bastano due che rendano ``?'' tutti gli attributi tranne il secondo,
  che deve restare 0. Prendo quindi un esempio positivo che, tranne per il secondo
  valore che deve restare 0 e per un'altro dei tre che deve restare come per
  $x_1$ in modo da avere un ulteriore esempio positivo, sia il complemento del
  primo. Ho quindi $x_2=(1, 0, 0, 1)$, con 
  $t(x_2)=1$. Ho già ottenuto:
  \[S=\{\langle 1, 0,?,?\rangle\}\]
  Ora, per il terzo esempio, prendo $x_3=(0, 0, 0, 1)$, con $t(x_3)=1$. Ottengo
  esattamente:
  \[S=\{\langle ?, 0,?,?\rangle\}\]
  Gli altri due esempi possono essere con qualsiasi valori ma devono essere
  negativi, non influenzando il risultato.\\
  \begin{shaded}
    Alternativa:
    \begin{itemize}
      \item $x_1=(1, 0, 1, 0)$ con $t(x_1)=1$
      \item $x_2=(0, 1, 0, 0)$ con $t(x_2)=0$
      \item $x_3=(1, 0, 1, 1)$ con $t(x_3)=1$
      \item $x_4=(0, 0, 0, 0)$ con $t(x_4)=0$
      \item $x_5=(0, 0, 0, 0)$ con $t(x_5)=1$
    \end{itemize}
    Con gli ultimi due che rappresentano, in un contesto reale,
    un \textbf{errore} e un \textbf{rumore}, in quanto gli stessi valori portano
    ad avere le due \textit{label} opposte, si ha quindi inconsistenza. Sono
    infatti detti \textbf{esempi inconsistenti} e vengono completamente ignorati
    da \textbf{find-S}.
  \end{shaded}
\end{esempio}
Sono evidenti i problemi di \textbf{find-S}:
\begin{itemize}
  \item Non si ha la garanzia che l'ipotesi in output sia l'unica consistente per il dato training set.
  \item I training set inconsistenti possono ingannare l'algoritmo.
\end{itemize}
D'altro canto:
\begin{itemize}
  \item Garantisce che l'ipotesi in output sia quella piu specifica
  e consistente per gli esempi positivi del training set.
  \item L'ipotesi finale è consistente anche con gli esempi negativi, a patto
  che il \textit{target concept} sia contenuto in $H$ e gli esempi siano
  corretti.
\end{itemize}
\begin{esempio}
  Siano dati due attributi $A=\{1, 2, 3\}$ e $B=\{1, 2\}$. Diciamo che $H$ è un
  congiunzione di $and$ tra i valori degli attributi e delle istanze.\\
  Il concetto target è:
  \[c:=((A=1\lor A=2), B=1)\]
  Find-S può trovare $c$ in $H$?\\
  Prendo un training set contenente $\langle x_1=(1, 1), 1\rangle$ e
  $x_2=\langle(2, 1), 1\rangle$. Seguendo find-S 
  avremo la seguente traccia dell'algoritmo:
  \begin{table}[H]
    \centering
    \begin{tabular}{c|c}
      A & B\\
      \hline
      $\emptyset$ & $\emptyset$\\
      1 & 1 \\
      ? & 1
    \end{tabular}
  \end{table}
  Si arriva quindi a:
  \[S=\langle ?, 1\rangle\]
  Ma questa generalizzazione mi farebbe accettare $A=3$, cosa non prevista da
  $c$.\\
  Non posso quindi usare find-S in questo caso.
\end{esempio}
\subsection{Algoritmi di eliminazione - Ricerca a impegno minimo}
Un'alternativa sarebbe quella di mantenere in memoria tutte e sole le ipotesi che sono consistenti con i dati raccolti. Ogni nuove istanza non avrà alcun effetto, oppure escluderà alcune delle ipotesi. L'algoritmo a \textbf{ricerca a impegno minimo} consiste nel mantenere nello spazio delle ipotesi solo quelle che rimangono consistenti mentre le altre vengono eliminate. L'insieme rimanente (contenente quindi l'ipotesi corretta) viene denominato \textbf{version space} o, analogamente, \textbf{spazio delle versioni}. Esso è uguale alla differenza tra lo \textbf{spazio delle ipotesi} e l'insieme delle ipotesi inconsistenti.
\begin{definizione}
  Si definisce \textbf{version space}, rispetto ad $H$ e $D$, il
  sottoinsieme delle ipotesi da $H$ consistenti con $D$ e si indica con:
  \[VS_{H, D}=\{h\in H|\, h(x) = c(x)\,\,\, \forall x \in D\}\]
\end{definizione}
Questo approccio ha la proprietà importante di essere \textit{incrementale}: non si deve mai tornare indietro per riesaminare vecchi esempi, dato che è comunque garantito che tutte le ipotesi rimanenti siano consistenti con essi. Voi cari lettori amanti dell'intelligenza artificiale ora mi chiederete: \textit{"Mario, ma come facciamo a rappresentare primariamente l'intero spazio delle ipotesi?"}. La domanda è abbastanza lecita, difatti su $n$ attributi potremmo avere un numero d'ipotesi semanticamente differenti pari a: 
 \[|H|_{sem}=1+\prod_{i}(|n_i|+1)\]
 La soluzione sta nell'usare una rappresentazione a intervallo che specifica solamente gli estremi dell'insieme. Questo funziona perché sullo spazio delle ipotesi è presente un ordinamento \textbf{parziale} (quello relativo alle relazioni di generalizzazione/specializzazione). Essendo un ordinamento parziale gli estremi verranno rappresentati da insiemi d'ipotesi, chiamati \textbf{insieme limitati}. Il vantaggio di tale pratica risiede nella rappresentazione dell'intero insieme d'ipotesi mediante due soli insiemi limitati: il più generale (\textbf{G-insieme}) e quello più specifico (\textbf{S-insieme}). Tale descrizione preserva la consistenza di tutte le ipotesi che si trovano tra le estremità. Sfortunatamente penso di aver risposto parzialmente alla vostra domanda: come inizializziamo il tutto? \\ L'inizializzazione dello spazio delle versioni iniziali (prima di considerare un qualsiasi esempio) è data dall'inizializzazione dell'insieme $G$ a \textbf{1}, mentre $S$ a \textbf{0}.\\
 Sperando di aver chiarito alcuni dei vostri dubbi (e sperando di avervene fatto rincarare molti altri), mi aspetterei da voi la seguente domanda: \textit{"Bene Mario, ora sappiamo come strutturare e inizializzare lo spazio delle versioni, ma come facciamo a modificarlo a run-time?}. La risposta alla domanda è banale, ciò di cui ci dobbiamo occupare sono gli elementi (che sono ipotesi) $S_i$ e $G_i$ degli insiemi $S$ e $G$. Sfruttiamo allora l'ideologia, precedentemente descritta, di falso positivo e falso negativo:
 \begin{itemize}
     \item Falso positivo per $S_i$: Questo significa che $S_i$ è troppo generale poiché riconosce un esempio in più rispetto al DataSet. Quindi dovremmo rendere più specifica questa ipotesi, ma per ipotesi essa è già la più specifica: quindi la eliminiamo dall'insieme $S$.
     \item Falso negativo per $S_i$: troppo specifico, quindi lo sostituiamo con tutte le sue immediate generalizzazioni, a patto che siano più specifiche di qualche elemento di $G$.
     \item Falso positivo per $G_i$: troppo generale, per cui lo rimpiazziamo con tutte le sue immediate specializzazioni, a patto che siano più generali di qualche elemento di $S$.
     \item Falso negativo per $G_i$: troppo specifico, lo scartiamo.
 \end{itemize}
 Queste operazioni proseguono finché non ci troviamo in uno dei seguenti casi:
 \begin{itemize}
     \item Rimane un'unica ipotesi nello spazio delle versioni.
     \item Lo spazio delle ipotesi collassa poiché $G$ e $S$ sono vuoti.
     \item Gli esempi finiscono e rimangono più ipotesi.
 \end{itemize}
    Vediamo quindi algoritmo \textbf{candidate eliminate}
\begin{algorithm}[H]
  \begin{algorithmic}
    \Function{CE}{}
    \State $G\gets$ \textit{insieme delle ipotesi più generali in $H$}
    \State $S\gets$ \textit{insieme delle ipotesi più specifiche in $H$}
    \For {\textit{ogni esempio di training $d=\langle x, c(x)\rangle$}}
    \If {\textit{d è un esempio positivo}}
    \State \textit{rimuovi da $G$ ogni ipotesi inconsistente con $d$}
    \For {\textit{ogni ipotesi $s$ in $S$ inconsistente con $d$}}
    \State \textit{rimuovi $s$ da $S$}
    \State
    \State \textit{aggiungi a $S$ tutte le generalizzazioni minime $h$ di $s$}
    \State \textit{tali che $h$ sia consistente con $d$ e qualche membro di $G$}
    \State \textit{sia più generale di $h$}
    \EndFor
    \State \textit{rimuovi da $S$ ogni ipotesi più generale di un'altra in $S$}
    \Else
    \State \textit{rimuovi da $S$ ogni ipotesi inconsistente con $d$}
    \For {\textit{ogni ipotesi $g$ in $G$ inconsistente con $d$}}
    \State \textit{rimuovi $g$ da $G$}
    \State
    \State \textit{aggiungi a $G$ tutte le generalizzazioni minime $h$ di $g$}
    \State \textit{tali che $h$ sia consistente con $d$ e qualche membro di $S$}
    \State \textit{sia più generale di $h$}
    \EndFor
    \State \textit{rimuovi da $G$ ogni ipotesi più generale di un'altra in $G$}
    \EndIf
    \EndFor
    \Return \textit{la lista delle ipotesi in $vs$}
    \EndFunction
  \end{algorithmic}
  \caption{Algoritmo Candidate Eliminate}
\end{algorithm}
Questo algoritmo ha alcune proprietà:
\begin{itemize}
  \item Converge all'ipotesi $h$ corretta provando che non ci sono errori in $D$
  e che $c\in H$.
  \item Se $D$ contiene errori allora l'ipotesi corretta sarà eliminata dal
  \textit{version space}.
  \item Si possono apprendere \textbf{solo} le congiunzioni.
  \item Se $H$ non contiene il concetto corretto $c$, verrà trovata l'ipotesi
  vuota.
  \end{itemize}
  Ci sono anche numerose limitazioni che pone l'algoritmo:
  \begin{itemize}
  \item Se il dominio di collisione contiene rumore o un numero di attributi insufficiente per una classificazione esatta, lo spazio delle versioni collasserà sempre.
  \item Se non poniamo alcun limite alle disgiunzioni nello spazio delle ipotesi, l'insieme $S$ conterrà sempre una singola ipotesi più specifica ovvero quella relativa alla descrizione degli esempi positivi visti a run-time. In modo analogo l'insieme $G$ conterrà solo la negazione della disgiunzione della descrizione degli esempi negativi.
  \item Per alcuni spazi delle ipotesi il numero di elementi degli insiemi $S$ e $G$ potrebbe crescere esponenzialmente con il numero di attributi.
\end{itemize}
Il nostro spazio delle ipotesi non è in grado di rappresentare un semplice
concetto di target disgiuntivo, si parla infatti di \textbf{Biased Hypothesis
  Space}. \\
Studiamo quindi un \textbf{unbiased learner}.\\Si vuole scegliere un $H$ che
esprime ogni concetto insegnabile, ciò significa che $H$ è l'insieme di tutti i
possibili sottoinsiemi di $X$. $H$ sicuramente contiene il concetto target. $S$
diventa l'unione degli esempi positivi e $G$ la negazione dell'unione di quelli
negativi. Per apprendere il concetto di target bisognerebbe presentare ogni
singola istanza in $X$ come esempio di training.\\
\subsubsection{Considerazioni sul learner}
Un learner che non fa assunzioni a priori in merito al concetto target non ha
basi ``razionali'' per classificare istanze che non vede.\\
\begin{itemize}
  \item Un algoritmo di learning del concetto $L$.
  \item Degli esempi di training $D_C=\{\langle x, c(x)\rangle\}$.
\end{itemize}
Si ha che $L(x_i, D_c$) denota la classificazione assegnata all'istanza $x_1$, da
$L$, dopo il training con $D_c$.
Possiamo quindi distinguere:
\begin{itemize}
  \item \textbf{Sistema induttivo}: si hanno in input gli esempi di
  training e la nuova istanza; viene usato l'algoritmo \textit{candidate
    eliminate} con $H$ e si ottiene la classificazione della nuova istanza
  nulla.
  \item \textbf{Sistema deduttivo}: equivalente al sistema induttivo sopra
  descritto dove in input si aggiunge l'asserzione ``$H$ contiene il concetto
  target'' e si produce lo stesso output tramite un \textbf{prover di teoremi}.
\end{itemize}
Abbiamo quindi visto tre tipi di \textit{learner}:
\begin{enumerate}
  \item Il \textbf{rote learner}: si ha classificazione sse $x$ corrisponde
  a un esempio osservato precedentemente. Non si ha \textit{bias induttivo}.
  \item L'algoritmo \textbf{candidare eliminate} con \textbf{version space}:
  il \textit{bias} corrisponde al fatto che lo spazio delle ipotesi contiene il concetto target. 
  \item L'algoritmo \textbf{Find-S}: il \textit{bias} corrisponde al fatto
  che lo spazio delle ipotesi contiene il concetto target e tutte le istanze
  sono negative a meno che il target opposto sia implicato in un altro modo. 
\end{enumerate}
\newpage
%----------------------------------------------------------------------------------------
%	SECTION 
%----------------------------------------------------------------------------------------